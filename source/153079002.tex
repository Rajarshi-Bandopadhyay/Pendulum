% Latex file for PDFLatex, prepared by Rajarshi Bandopadhyay.

\documentclass{article}
\usepackage{graphicx}
\usepackage[pdftex,
            pdfauthor={Rajarshi Bandopadhyay},
            pdftitle={Pendulum with Friction: Plots of Key Parameters},
            pdfsubject={Pendulum With Friction},
            pdfkeywords={Oscillation, Transients, Pendulum, Friction},
            pdfproducer={pdflatex},
            pdfcreator={pdflatex}]{hyperref}

\begin{document}

\pdfinfo{
   /Author (Rajarshi Bandopadhyay)
   /Title  (Pendulum with Friction: Plots of Key Parameters)
   /CreationDate (D:20040502195600)
   /Subject (Pendulum With Friction)
   /Keywords (Oscillation;Transients;Pendulum;Friction)
}

\begin{Large} Pendulum with Friction: plots of key parameters \end{Large}
\begin{flushright}
- by Rajarshi Bandopadhyay
\end{flushright}

\paragraph{}
This project has been made possible by the detailed study of a pendulum under the effect of friction, as performed in \cite{theory}. The aim of the project is to generate plots of two key parameters, namely the instantaneous angle $\theta$ and the instantaneous angular velocity $\omega$, against time as an independent parameter. \\
These parameters were chosen based on the lectures in \cite{feynman}.

\paragraph{}
The project is saved on a GitHub repository at the following address: \\
\textit{https://github.com/Rajarshi-Bandopadhyay/Pendulum}. \\
The Python files must be run using Python-3. \\
Parameters used: 

\paragraph{Animations:}
The ipython notebooks named \textit{animate\_theta} and \textit{animate\_omega} have been used for generating animations. Both require the use of the \textit{theory.py} module, which contains the definition of the \textit{FrictionPendulum} class.

\paragraph{Plots}
Due to constraints of space, the plots are shown on the page that follows.

\pagebreak
\begin{figure}[t]
\title{Plots obtained:}

\includegraphics[scale=0.30]{theta.png} 
\caption{Angle of the pendulum $\theta$ vs time.}

\includegraphics[scale=0.30]{omega.png}
\caption{Angular velocity of the pendulum $\omega$ vs time.}
\end{figure}

\begin{tabular}{c}
 \\
 \\
 \\
 \\
 \\
\end{tabular}
\bibliographystyle{plain}
\bibliography{references.bib}

\end{document}